\chapter{Trabajos futuros}\label{cap6}

Tras terminar el proyecto hay muchas mejoras que han quedado pendientes o sería deseable poder añadir. A continuación puede verse una lista de aquellas elementos que se han considerado más importante de cara a la continuidad del proyecto:

\begin{description}
	\item[Mejorar las comunicaciones entre agentes y controlador.]

		Hasta este momento las comunicaciones entre agente y controlador hacen uso de un protocolo un poco pobre en lo que a características se refiere lo que supone también una restricción a la hora de hacer ampliaciones. La mejora propuesta busca ampliar este protocolo para que la variedad de funciones que pueda tener el agente sea mucho mayor.

		Para desarrollar el nuevo protocolo habría que:

		\begin{itemize}
			\item Estudiar las mejoras que se quieren introducir para saber cómo afectarían a las comunicaciones (por ejemplo, para enviar \emph{plugins} hay que poder indicarlo).

			\item Definir el nuevo protocolo (mensajes que envia el agente al controlador y las posibles respuestas a estos) con respecto al punto anterior.

			\item Realizar las modificaciones pertinenten en el controlador y en el agente. Para ello habrá que cambiar el fichero \emph{agents\_controller.php}, del controlador, y \emph{ControllerLink.cpp}, del agente.
		\end{itemize}

	\item[Carga de algoritmos desde el controlador.]

		Esta extensión permitiría centralizar la administración de los algoritmos en el controlador de modo que no haga falta tener que entrar de forma remota en cada uno de los agentes para añadir nuevas funcionalidades. Esto facilitaría la administración del sistema al evitar tener que acceder a cada agente para instalar las extensiones.

		Para realizar esta tarea habría que modificar el controlador y el agente de tal modo que:

		\begin{itemize}
			\item El controlador debe saber en todo momento que algoritmos y \emph{executors} tienen los agentes.

			\item El controlador pueda alojar los nuevos \emph{plugins}.

			\item El controlador pueda enviar a los agentes los \emph{plugins}.

			\item El Agente debe poder cargar los \emph{plugins} cuando se lo indique el controlador.
		\end{itemize}

		Para esto habría que cambiar el protocolo de comunicación que se utiliza actualmente.

	\item[Crear nuevos algoritmos de comprobación de hashes.]

		En la actualidad hay una gran cantidad de algoritmos de resumen que pueden ser portados a  DHC y sería interesante disponer de ellos.

		Estos algoritmos se pueden crear fácilmente dentro de \emph{plugins} que serán cargados posteriormente. Para ello cada nuevo algoritmo deberá heredar de la clase \emph{Algorithm}, implementar su funcionalidad y ser exportado como \emph{plugin}.

	\item[Crear nuevos mecanismos de ejecución.]

		En estos momentos los algoritmos solo pueden hacer uso de un sistema básico de ejecución, pero podrían implementarse nuevos sistemas que permitiesen mejoras en los tiempos o simplemente implementar nuevos algoritmos más allá de las funciones resumen.

		Estos nuevos mecanismos de ejecución pueden implementarse de distintos modos dependiendo del objetivo. Por ejemplo, en el caso de querer realizar optimizaciones sobre el sistema existente actualmente habría que:

		\begin{itemize}
			\item Estudiar el \emph{Executor} que se quiere mejorar para buscar sus cuellos de botella, variables innecesarias, etc.

			\item Crear un nuevo \emph{Executor} con las modificaciones que implementen las mejoras.

			\item Probar los cambios realizados.
		\end{itemize}

		En caso de querer crear un nuevo \emph{Executor} para una funcionalidad aún no implementada, los pasos serán los mismos, pero hay que estudiar como deberá ejecutarse la nueva funcionalidad.

	\item[Implementar nuevos protocolos de seguridad.]

		DHC no tiene porque restringirse solo a funciones resumen. Con esto se conseguiría que la herramienta abarcase una mayor número de mecanismos de seguridad como  podrían ser:

		\begin{itemize}
			\item Contraseñas de redes WiFi con protección WPA.

			\item Determinar las claves generadas a partir de un \emph{handshake} del sistema TLS utilizado en páginas web.
		\end{itemize}

		Para este propósito habrá que hacer uso de los puntos anteriores ya que puede que se deba cambiar el protocolo de comunicaciones además de añadir nuevos algoritmos y \emph{executors}.

	\item[Control de cambios sobre el sistema de ficheros.]

		Esta característica puede utilizarse para determinar cuando un \emph{plugin} ha cambiado y volverlo a cargar sin tener que reiniciar el agente. De este modo el rendimiento de la aplicación mejoraría al no tener que parar casi nunca.

		El desarrollo de esta mejor podría hacerse del siguiente modo:

		\begin{itemize}
			\item Se debe determinar las llamadas al sistema que permiten monitorizar los cambios sobre el sistema de ficheros.

			\item Hay que diseñar un mecanismo que permita eliminar un \emph{plugin} de la memoria para ser sustituido por otro. Hay que tener en cuenta que esto supone saber si está o no en uso y que cuando se realice el cambio no podrán ser utilizados.

			\item Hay que desarrollar y probar la solución propuesta.
		\end{itemize}

	\item[Sistema de controladores jerárquicos.]

		El objetivo de este cambio es poder disponer de un árbol de controladores donde los que se encuentren en las ramas sean los encargados de repartir el trabajo entre los agentes y los demás controladores simplemente repartan trabajo entre ellos. El objetivo es repartir la carga en casos en los que haya una gran cantidad de agentes.

		Este sistema necesita que se cambien los controladores, de tal modo que puedan comunicarse también entre ellos. Hay varias opciones para conseguir esto:

		\begin{itemize}
			\item Crear dicha funcionalidad y ejecutarla desde un planificador de tareas.

			\item Esperar un número de peticiones de los agentes para realizar la llamada al controlador superior. Esta llamada se realizaría durante la solicitud del agente ya que PHP no puede crear tareas en paralelo para este tipo de acciones.
		\end{itemize}

		La elección de una u otra solución dependerá del uso que se vaya a dar al sistema, por lo que habrá que estudiar como afectaría al rendimiento cada caso.
\end{description}