\chapter{Trabajos futuros}\label{cap6}

Tras terminar el proyecto hay muchas mejoras que han quedado pendientes o sería deseable poder añadir. A continuación puede verse una lista de aquellas elementos que se han considerado más importante de cara a la continuidad del proyecto:

\begin{description}
	\item[Carga de algoritmos desde el controlador] Esta extensión permitiría centralizar la administración de los algoritmos en el controlador de modo que no haga falta tener que entrar de forma remota en cada uno de los agentes para añadir nuevas funcionalidades. Esto facilitaría la administración del sistema al evitar tener que acceder a cada agente para instalar las extensiones.

	\item[Mejorar las comunicaciones entre agentes y controlador] Hasta este momento las comunicaciones entre agente y controlador hacen uso de un protocolo un poco pobre en lo que a características se refiere lo que supone también una restricción a la hora de hacer ampliaciones. La mejora propuesta busca ampliar este protocolo para que la variedad de funciones que pueda tener el agente sea mucho mayor.

	\item[Crear nuevos algoritmos de comprobación de hashes] En la actualidad hay una gran cantidad de algoritmos de resumen que pueden ser portados a  DHC y sería interesante disponer de ellos.
	
	\item[Crear nuevos mecanismos de ejecución] En estos momentos los algoritmos solo pueden hacer uso de un sistema básico de ejecución, pero podrían implementarse nuevos sistemas que permitiesen mejoras en los tiempos o simplemente implementar nuevos algoritmos más allá de las funciones resumen.
	
	\item[Implementar nuevos protocolos de seguridad] DHC no tiene porque restringirse solo a funciones resumen. Con esto se conseguiría que la herramienta abarcase una mayor número de mecanismos de seguridad como  podrían ser:
	
	\begin{itemize}
		\item Contraseñas de redes WiFi con protección WPA.
		
		\item Determinar las claves generadas a partir de un \emph{handshake} del sistema TLS utilizado en páginas web.
	\end{itemize}

	\item[Control de cambios sobre el sistema de ficheros] Esta característica puede utilizarse para determinar cuando un plugin ha cambiado y volverlo a cargar sin tener que reiniciar el agente. De este modo el rendimiento de la aplicación mejoraría al no tener que parar casi nunca.
	
	\item[Sistema de controladores jerárquicos] El objetivo de este cambio es poder disponer de un árbol de controladores donde los que se encuentren en las ramas sean los encargados de repartir el trabajo entre los agentes y los demás controladores simplemente repartan trabajo entre ellos. El objetivo es repartir la carga en casos en los que haya una gran cantidad de agentes.
\end{description}