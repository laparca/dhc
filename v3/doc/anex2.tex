\chapter{Entorno de desarrollo} 
\section{Instalación del entorno}

Una vez que se conoce como es la arquitectura, será necesario instalar el entorno CUDA en el sistema para poder utilizarlo. NVIDIA nos provee de 3 partes distintas:

\begin{itemize}
	\item El controlador gráfico con soporte de CUDA que es necesario para poder aprovechar las capacidades de cómputo de la tarjeta gráfica.
	\item El SDK que contiene todas la bibliotecas necesarias para poder desarrollar aplicaciones con soporte de ejecución en GPU.
	\item El Toolkit con las herramientas necesarias para generar los programas (como nvcc, el compilador de código CUDA) además de bibliotecas de alto nivel como BLAS y aceleración de FFT (transformada rápida de Fourier).
\end{itemize}

En la web de NVIDIA recomiendan que se instalen los componentes en el mismo orden en el que aparecen en el listado anterior.

A la hora de realizar la instalación de CUDA es importante tener en cuenta la plataforma (Windows, MacOS X, GNU/Linux) y las versiones. Para el desarrollo del presente proyecto fin de carrera se ha utilizado como equipo de desarrollo un sistema con Ubuntu 10.04 y el equipo de ejecución utiliza Debian GNU/Linux en su última versión estable.