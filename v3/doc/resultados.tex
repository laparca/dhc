\chapter{Resultados}

En este capítulo se presentan las pruebas realizadas sobre el proyecto y los resultados obtenidos por las mismas.

Con las pruebas realizadas se ha procurado asegurar el buen funcionamiento de los cambios introducidos durante la realización del proyecto para garantizar su utilidad.

\section{Subsistema de \emph{plugins}}

Las pruebas sobre el sistema de \emph{plugins} tratan de comprobar, por una parte, el buen funcionamiento de la carga dinámica de las extensiones, así como de las mejoras suministradas por el API de algoritmos y de \emph{executors}.

Este apartado ha sido uno de los más importantes ya que todos los cambios realizados tienen un gran impacto sobre la estructura del proyecto. De este modo se debe garantizar que tras los cambios todo el sistema sigue funcionando de forma correcta.

Los pasos seguidos para el propósito descrito han sido el siguiente:

\begin{enumerate}
	\item Paso del algoritmo de MD4 al nuevo sistema.
	
	Con esto se pretendía probar si el mecanismo de control de algoritmos funcionaba correctamente sin influir en los resultados.
	
	Se eligió el algoritmo de MD4 al azar entre todos los implementados en DHC.
	
	\item Paso de todos los algoritmos antiguos al nuevo sistema.
	
	Así se garantizaba el buen funcionamiento en todos los casos y se podía comprobar que no hubiese código antiguo interfiriendo en en la ejecución.
	
	\item Creación de \emph{BasicExecutor}.
	
	Este \emph{executor} implementa la lógica antigua de funcionamiento de los algoritmos. Se pretende de este modo reducir la cantidad de código duplicado y comprobar a su vez el funcionamiento de este subsistema.
	
	Como en el caso anterior el primer algoritmo en hacer uso de esta funcionalidad ha sido MD4.
	
	\item Uso de \emph{BasicExecutor} en todos los algoritmos.
	
	Con este cambio se comprueba el funcionamiento real de todos los algoritmos del sistema.
	
	\item Creación del plugin \emph{DummyPlug}.
	
	Este plugin de prueba no realiza ninguna operación dentro del sistema, pero es útil para comprobar que el subsistema de \emph{plugins} es capaz de cargar correctamente los mismos.
	
	\item Transformación de los algoritmos en \emph{plugins}.
	
	Finalmente, se pasaron todos los algoritmos para que hagan uso del sistema de \emph{plugins} y así comprobar el funcionamiento real; de este modo se independizan los \emph{plugins} del agente.
\end{enumerate}

Para realizar las pruebas anteriores se fueron realizando los cambios paulatinamente y corrigiendo los errores que surgieron. En estos momentos se puede considerar que el subsistema de \emph{plugins}, algoritmos y \emph{executors} funciona correctamente.

\section{Controlador}

El controlador ha sufrido un rediseño importante de su aspecto que ha buscando mantener la misma funcionalidad que tenía anteriormente y a la vez ofrecer una experiencia más agradable mejorando la presentación. Para ello se ha hecho uso de las técnicas actuales de diseño web como uso de degradados.

Los cambios realizados han supuesto una mejora constatada tras dar a probar ambas interfaces a un grupo de personas. Este grupo se componía de 7 personas completamente ajenas al desarrollo de este proyecto y pudieron constatar la mejora del mismo.

Por otro lado, los cambios en los agentes destinados a mejorar el tiempo de respuesta del controlador han contribuido a que, en momentos de alta carga de trabajo, los tiempos de espera se reduzcan. Esto supone una mejor experiencia de usuario al eliminar posibles impaciencias por la tardanza en obtener una página del controlador. Para poder comprobar este último punto solo ha sido necesario enviar un hash al sistema para que se ponga en marcha y controlar el tiempo que tarda en devolver una página.

\section{Mantenibilidad}

La facilidad de mantener el sistema era un punto importante del proyecto por la falta de organización del mismo. Con los cambios que se han introducido se ha procurado mejorar este aspecto reduciendo el número de ficheros a cambiar para pequeños cambios.

Un ejemplo claro sería la introducción de un nuevo algoritmo, que antes suponía la modificación de al menos los siguientes ficheros:

\begin{itemize}
	\item ControllerLink.cpp
	\item ComputeThreadProc.cpp
\end{itemize}

Además, puede ser necesario el cambio de más ficheros dependiendo de cómo se haya implementado el algoritmo.

Tras la reimplantación del sistema se simplifica enormemente el número de ficheros a modificar ya que ahora solo hace falta cambiar el del propio algoritmo. Igualmente, al añadir nuevas funcionalidades se simplifica enormemente ya que no hay necesidad de modificar línea alguna de código sobre el agente.

Igualmente el controlador se ha visto claramente beneficiado por el uso de CakePHP ya que, gracias al uso de una estructura ordenada en la organización de los elementos que componen el software, se puede ir directamente a la parte en la que pueda haber cualquier problema para solucionarla.

\section{Algoritmo SHA-256}

Las pruebas sobre este algoritmo han consistido, por una parte, en comprobar que los resúmenes que generaba eran correctos. Una vez confirmado este punto se ha procedido a calcular el tiempo que tarda en calcular un resumen para poder compararlo con el resto de algoritmos.

Para comprobar el tiempo que tarda se eligió una palabra de prueba con minúsculas, mayúsculas, números y algunos símbolos y una longitud de 8 caracteres. Tras esto se generó su resumen SHA-1 y SHA-256. Tras medir los tiempos que tardaba el sistema en encontrar la palabra para SHA-1 y SHA-256 se pudo determinar que la implementación de este último es 1,5 veces más lenta que la primera. Esto se debe principalmente a que el nuevo algoritmo requiere más pasos para obtener el resumen.

El resultado anterior es importante de cara a la implementación de nuevos algoritmos ya que será importante tener en cuenta cuanto más complejos son y, en caso de una complejidad excesiva, si es útil implementarlo. Si la complejidad de un algoritmo resultase muy elevada podría suceder que el tiempo que tardase en devolver un resultado fuera demasiado elevado. Esto significa que es importante comprobar con las ecuaciones mostradas en~\ref{sec:comprobacion_resumen} para determinar de forma anticipada los tiempos.