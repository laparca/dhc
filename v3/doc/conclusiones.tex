\chapter{Conclusiones}

Tras la finalización del proyecto y teniendo en cuenta los resultados podemos concluir que todas las modificaciones aportadas han permitido mejorar sustancialmente la herramienta, pudiendo destacarse:

\begin{itemize}
	\item La mejora en la mantenibilidad puede considerarse muy importante tras haberse simplificado de forma importante los puntos más críticos de la herramienta.
	
	\item El susbsistema de plugins permite simplificar el código del agente, separando la funcionalidad de control del mismo de las tareas más específicas que se le soliciten. De este modo DHC se convierte en una herramienta versatil a la que puede dotarsele de funcionalidades para las cuales no había sido diseñado inicialmente.
	
	\item El nuevo diseño convierte al DHC en un sistema atractivo e intuitivo que anima a ser utilizado lo que favorece su continuidad en el tiempo frente a otras posibles soluciones.
	%explicar este punto bien en la presentación
	
	\item Tras el estudio y uso de la herramienta se puede concluir que es muy importante la calidad de las contraseñas utilizadas ya que en su fortaleza depende una parte importante de la seguridad de muchas instituciones.
	
	\item El uso de \emph{frameworks} de MVC facilita en gran medida el desarrollo de aplicaciones web al eliminar la necesidad de preocuparse por detalles de bajo nivel.
	
	\item Las herramientas de control de versiones simplifican enormemente la gestión del código de los proyectos y facilitan las tareas de vuelta atrás en el tiempo en caso de fallos graves.
\end{itemize}

Finalmente cabe concluir que tras el desarrollo de este proyecto se ha podido comprobar las dificultades que entraña el mantenimiento de herramientas con una escasa documentación.