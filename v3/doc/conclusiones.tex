\chapter{Conclusiones}

EL proyecto ha terminado existosamente tras haber realizado toda una serie de actividades que han permitido organizar y realizar el trabajo deseado. Estas actividades fueron planteadas cuando se inicio el proyecto y garantizan la buena organización del proyecto.

A continuación puede verse las actividades realizadas ordenadas según se realizaron:

\begin{itemize}
	\item Búsqueda de soluciones existentes.

		El primer paso que se realizó fue buscar si existía alguna herramienta que, siendo sofwate libre, permitiese realizar la evaluación de contraseñas utilizando tarjetas gráficas. Solo se encontró una, DHC.

	\item Estudio de la solución elegida.

		Disponiendo de una herramienta, se procedió a estudiar el código fuente de ésta para poder determinar como añadir nuevas funciones resumen y como extender las funcionalidades que éste ya tenía.

		Desgraciadamente, la documentación que tenía DHC abarcaba principalmente el proceso de compilación y ejecución, pero no detallaba cómo funcionaba por dentro por lo que se tuvo que hacer un gran esfuerzo en su estudio.

	\item Determinar las mejoras a realizar.

		Conociendo el código se podía empezar a proponer mejoras sobre el mismo. Las mejoras debían en todo momento garantizar que la herramienta se adapte completamente a los requisitos propuestos (facilidad de ampliación, fácilidad de mantenimiento, agilidad de uso, etc.).

		Las mejoras propuestas para DHC ha sido el núcleo de este proyecto y que ya se han visto anteriormente.

	\item Llevar a cabo las mejoras propuestas.

		A lo largo del capítulo~\ref{cap:mejoras} se ha podido ver el desarrollo de las mejoras propuestas, en qué han consistido y cómo se han llevado a cabo.

	\item Pruebas de aceptación.

		Para garantizar el buen funcionamiento de las modificaciones realizadas se ha tenido que comprobar que los resultados generados fueran los correctos
\end{itemize}




Tras la finalización del proyecto, y teniendo en cuenta los resultados, podemos concluir que todas las modificaciones aportadas han permitido mejorar sustancialmente la herramienta, pudiendo destacarse:

\begin{itemize}
	\item La mejora en la mantenibilidad puede considerarse muy importante tras haberse simplificado de forma importante los puntos más críticos de la herramienta. Esto supone mayor facilidad para realizar cambios en caso de errores o de tener que introducir nuevas mejoras.
	
	\item Cuando se tiene un código que hace un uso intensivo de sentencias \emph{if} anidadas este puede, en ciertos casos, ser sustituido por mecanismos de herencia. En el caso concreto del proyecto, esto ha permitido incorporar todo el sistema de algoritmos en sustitución de un mecanismo estático y complicado de modificar.
	
	\item El subsistema de \emph{plugins} permite simplificar el código del agente, separando la funcionalidad de control del mismo de las tareas más específicas que se le soliciten. De este modo DHC se convierte en una herramienta más versátil a la que se le puede dotar de nuevas funcionalidades para las cuales no había sido diseñado inicialmente.
	
	\item El uso del CakePHP ha permitido simplificar el código del controlador gracias al uso de las herramientas que éste nos ofrece. Además, esto facilita enormemente el realizar cambios sobre la interfaz y mantener una mayor organización código.
	
	\item El uso de \emph{frameworks} MVC facilita en gran medida el desarrollo de aplicaciones web al eliminar la necesidad de preocuparse por detalles de bajo nivel como puedan ser el acceso a base de datos.

	\item El nuevo diseño creado para DHC lo convierte en un sistema más atractivo, lo que anima a que sea utilizado. Este punto es importante ya que generalmente la gente se siente más agusto con aquello que encuentra agradable frente a soluciones que puedan ser más completas.
	%explicar este punto bien en la presentación
	
	\item Tras el estudio y el uso de la herramienta se puede concluir que es muy importante la calidad de las contraseñas utilizadas ya que en su fortaleza depende una parte importante de la seguridad de muchas instituciones. Igualmente, el tipo de función resumen es también importante ya que las debilidades que ésta pueda tener puede afectar enormemente a la integridad de los datos.
		
	\item Las herramientas de control de versiones simplifican enormemente la gestión del código de los proyectos ya que eliminan la necesidad de mantener las versiones manualmente. Además, al funcionar este tipo de herramientas en red permite disponer del código en cualquier parte eliminando la necesidad de tener que recordar el llevar una copia encima. Por otra parte, también facilitan las tareas de vuelta atrás en el tiempo en caso de fallos graves, de este modo en caso de cometer un error se puede ir fácilmente a una versión anterior para deshacer los cambios.
\end{itemize}

Finalmente cabe concluir que tras el desarrollo de este proyecto se ha podido comprobar las dificultades que entraña el mantenimiento de herramientas con una escasa documentación.