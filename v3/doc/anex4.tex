\chapter{Glosario}

\begin{description}
	\item[Algoritmo] Para el propósito de este proyecto un algoritmo es cualquier código destinado a poder ejecutarse en CPU o GPU con el fin de obtener una clave a partir de un conjunto de datos cifrados.
	
	\item[API] Application Programming Interface o interfaz de programación de aplicación. Es un conjunto de especificaciones destinadas a facilitar la realización de ciertas tareas.
	
	\item[Framework] Conjunto de herramientas y especificaciones que permiten desarrollar aplicaciones de forma sencilla. La diferencia con el API es que esta última solo define bibliotecas para utilizar, mientras que el framework puede ir acompañado de herramientas.
	
	\item[Makefile] Es un fichero que tiene las reglas a seguir para construir (compilar) un proyecto software.
	
	\item[MVC] Patrón Modelo Vista Controlador. Es una forma de organizar el código de tal forma que se diferencia claramente qué se encarga de la persistencia de datos (el modelo), qué se encarga del control de la lógica del programa (el controlador) y qué se encarga de presentar los datos al usuario (la vista).
\end{description}
