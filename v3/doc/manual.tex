\chapter{Manual de usuario} 

En el este capítulo se va a mostrar como instalar y usar DHC. Además se verán algunos detalles más avanzados de configuración para casos especiales.

\section{Instalación}

Antes de poder hacer uso de la aplicación es necesario instalarla. En las siguiente subsecciones se explicarán los pasos detallados para poner a punto el sistema.

\subsection{Instalación del agente}

El agente es uno de los elementos esenciales de DHC y el que entraña mayor dificultad de instalación. Por ese motivo hay que prestar especialdetalle al procedimiento para identificar posibles fallos.

El proceso de instalacion consiste en la instalación del sistema operativo en la máquina, los controladores del mismo y las herramientas de compilación tal y como se ha descrito en el apéndice~\ref{chap:entorno}.

Una vez instalado todas las herramientas necesarias se procede a la compilación del agente. Para este propósito se deberán seguir los siguiente pasos:

\begin{enumerate}
	\item Se hace una copia del repositorio para trabajar con ella:
	
	\begin{verbatim}
	$ git clone $LOCATION dhc
	\end{verbatim}
	
	\$LOCATION hace referencia al lugar en el que se encuentre el repositorio y puede ser de la forma \emph{file://directorio} o \emph{usuario@ssh://direccion/ruta}. Una vez realizada esta copia dispondremos de una version local del respositorio sobre la que trabajar. Esta versión se habrá creado en el directorio en el que nos estuviéramos en el momento de realizar la llamada.
	
	\item Preparamos la compilación:
	
	\begin{verbatim}
	$ cd dhc/v3
	$ cmake .
	$ mkdir ptx
	\end{verbatim}
	
	\item Compilamos el código:
	
	\begin{verbatim}
	$ make
	\end{verbatim}
	
	\item Una vez compilado el código disponemos en el directorio del proyecto de una carpeta llamda bin y otra ptx que contienen los binarios de linux y de CUDA.
\end{enumerate}

\begin{table}
	\centering
	
	\begin{tabular}{|c|p{5.6cm}|p{5.6cm}|}
	\hline
	Hecho & Descripción & Notas\\
	\hline
	& Instalar S.O. Linux en el ordenador & \\
	\hline
	& Instalar las herramientas de desarrollo de g++ y cmake & \\
	\hline
	& Comprobar e instalar, si es necesario, las cabeceras de Linux & \\
	\hline
	& Instalar controladores de NVIDIA con soporte de CUDA & \\
	\hline
	& Comprobar si se han creado los dispositivos de NVIDIA en /dev & \\
	\hline
	& Compilador el agente & \\
	\hline
	\end{tabular}
	
	\caption{Matriz de comprobación de la instalación del agente}\label{tab:mat_agente}
\end{table}

\subsection{Instalación de controlador}

\section{Administración de plugins}

\subsection{Instalación de nuevos algoritmos}

Esto podría apreciarse en el caso de que hubiese dos grupos de desarrollo y cada uno trabajase con una versión del agente diferente. Si cada agente debe contener los algoritmos que utiliza puede darse un caso como el siguiente: el agente del equipo 1 implementa los algoritmos A, B y C; el agente del equipo 2 tiene los algoritmos B y D. Así que si estamos utilizando el agente del equipo 1 y tenemos que probar el algoritmo D tenemos que deshacernos de los algoritmos A y C.

Separando la lógica de los algoritmos fuera del agente conseguimos mayor independencia. En el caso anterior solo tendríamos que instalar el nuevo algoritmo D sin tener que preocuparnos por los algoritmos A, B y C.

\subsection{Eliminación algoritmos obsoletos o erróneos}

Puede darse el caso de que un algoritmo pase a un estado de obsolescencia. Esto podría deberse a que se disponga de un algoritmo más nuevo que haga el mismo trabajo. También podría ser que tras probar un algoritmo descubramos que éste introduce errores en el sistema. En estos casos la mejor solución es eliminarlos con un simple borrado de disco.

Con el sistema tal y como venía era un problema solucionar esto, ya que se dependía del tiempo que pueda tardar el desarrollador en solucionar el problema.

\section{Uso del controlador}

\section{Configuraciones avanzadas}\label{sec:conf_avanzada}

\subsection{Varios controladores un MySQL}
Disponer de varios controladores gracias a que comparten una base de datos.

\subsection{Cambio de los tiempos de espera para evitar exceso de reciclado de WU}
En ciertas ocasiones, los agentes pueden tardar mucho en devolver el resultado de una WU por lo que se hace necesario modificar los tiempo de caducidad para evitar un exceso de reciclados que ralentizarían sobremedida el proceso de comprobación.