\chapter{Glosario}

\begin{description}
	\item[Algoritmo] Para el propósito de este proyecto un algoritmo es cualquier código destinado a poder ejecutarse en CPU o GPU con el fin de obtener una clave a partir de un conjunto de datos cifrados.
	
	\item[API] Aplication Programming Interface o interfaz de programación de aplicación. Es un conjunto de especificaciones destinadas a facilitar la realización de ciertas tareas.
	
	\item[Colision] Una colisión en el ámbito de las funciones resumen es cuando se dispone de dos mensajes diferentes que producen el mismo resumen.
	
	\item[Commit] Cada vez que se realizan modificaciones sobre fichero y se desea almacenar estas en el repositorio de versiones se realiza una operación de \emph{commit}.

	\item[CPU] \emph{Central Process Unit}. Es la unidad central de proceso y está encargada de ejecutar el código binario.
	
	\item[Distribución Linux] En el mundo del sistema operativo Linux se llama distribución a un paquete que contiene el núcleo Linux junto a un determinado conjunto de herramientas.
	
	\item[Fetch] Herramienta de git que permite consultar a un repositorio remoto por los cambios que han ocurrido. Estos cambios no se aplican localmente.
	
	\item[Framework] Conjunto de herramientas y especificaciones que permiten desarrollar aplicaciones de forma sencilla. La diferencia con el API es que esta última solo define bibliotecas para utilizar, mientras que el framework puede ir acompañado de herramientas.
	
	\item[GPU] \emph{Graphic Process Unit} o unidad de proceso gráfico es el procesador utilizado por una tarjeta gráfica para controlar las operaciones que se realizan sobre la misma. En la actualidad permiten la carga de programas de usuario especialmente diseñados.
	
	\item[HEAD] En git HEAD hace referencia a la última revisión que se encuentra en el repositorio.

	\item[Linux] Es un núcleo de sistema operativo que nació del núcleo Minix.
	
	\item[Makefile] Es un fichero que tiene las reglas a seguir para construir (compilar) un proyecto software.
	
	\item[MVC] Patrón arquitectónico Modelo-Vista-Controlador. Es un modo de organizar el código de tal forma que se diferencia claramente qué se encarga de la persistencia de datos (el modelo), qué se encarga del control de la lógica del programa (el controlador) y qué se encarga de presentar los datos al usuario (la vista).
	
	\item[Núcleo] En sistemas operativos el núcleo es un software que se encarga de facilitar los recursos del sistema a los programas.
	
	\item[Pull] En git se utiliza pull para aplicar a la rama actual de desarrollo que se encuentre activa los últimos cambios que se hayan hecho. Es recomendable hacer \emph{fetch} previamente.
	
	\item[Rack] Es un armario especial para poder poner equipos informáticos como ordenadores, switch, etc.
	
	\item[Rama de desarrollo] Es una línea de desarrollo de un proyecto en la que se hacen modificaciones. Las modificaciones realizadas en una rama no son vistas por el resto de ramas.
	
	\item[Push] Es el mandato que utiliza git para enviar todos los cambios locales a un repositorio remoto. Si no se indica ningún repositorio se utilizará el repositorio \emph{origin}.
	
%	\item[SLOC] \emph{Source Lines Of Code}. Son líneas de código fuente sin tener en cuenta los comentarios.
	
	\item[URL] Dirección de un elemento en internet. Suele estar definido por el tipo de protocolo, la dirección de la máquina y la ruta al elemento. Por ejemplo, http://www.google.com/reader. En este caso se accede a la máquina www.elpais.es utilizando el protocolo HTTP para solicitar /reader.
\end{description}
