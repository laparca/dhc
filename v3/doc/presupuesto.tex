\chapter{Presupuesto}

A continuación se va a presentar el presupuesto requerido para llevar a cabo el proyecto.

La duración total del proyecto ha sido de 11 meses, teniendo en cuenta que la dedicación que se ha aplicado ha sido de aproximadamente 13,2 horas semanales (un tercio de las 40 horas semanales).

\section{Desglose de actividades del proyecto}

Para calcular el total de horas del proyecto se ha tenido en cuenta todas las actividades realizadas en el mismo. En el cuadro~\ref{tab:des_horas} puede apreciarse el trabajo realizado, que ha ido desde el estudio de la solución previa, el análisis de los posibles cambios a realizar, el diseño de las modificaciones realizadas hasta la instalación del sistema de pruebas (la máquina Tesla).

\begin{table}
	\centering
	
	\begin{tabular}{|l|r|}
		\hline
		Actividad & Horas \\
		\hline
		Estudio DHC & 80 h \\
		\hline
		Anásis y diseño & 70 h \\
		\hline
		Implementación & 280 h \\
		\hline
		Pruebas & 40 h \\
		\hline
		Instalación sistema & 5 h \\
		\hline
		Documentación & 100 h \\
		\hline
	\end{tabular}
	\caption{Desglose de horas por actividad}\label{tab:des_horas}
\end{table}

\section{Gasto en personal imputable al proyecto}

Para ajustar el precio por hora del programador se ha hecho una búsqueda entre las ofertas de trabajo que se encuentran actualmente en internet. Se ha tomado como muestra aquellas en las que se busca profesionales con experiencia donde el sueldo ronda los 36.000\euro/año.

El coste total por personal es de 9.200\euro{} como puede apreciarse en el cuadro~\ref{tab:des_prec_horas}.

\begin{table}
	\centering
	
	\begin{tabular}{|l|r|r|r|}
		\hline
		Cargo & Horas & Coste/Hora & Total \\
		\hline
		Analista & 150 h & 16,00\euro/hora & 2.400,00\euro\\
		\hline
		Programador & 280 h & 16,00\euro/hora & 4.480,00\euro\\
		\hline
		Responsable documentación & 100h h & 16,00\euro/hora & 1.600,00\euro\\
		\hline
		Responsable pruebas & 40 h & 16,00\euro/hora & 640,00\euro\\
		\hline
		Técnico instalación & 5 h & 16,00\euro/hora & 80,00\euro\\
		\hline
		\hline
		Total & 575 h & \multicolumn{2}{|r|}{9.200,00\euro} \\
		\hline
	\end{tabular}
	\caption{Desglose de horas por actividad}\label{tab:des_prec_horas}
\end{table}


\section{Servicios subcontratados}

En el cuadro~\ref{tab:serv_subcont} puede apreciarse el coste de los servicios que se han subcontratado.

\begin{table}
	\centering
	
	\begin{tabular}{|l|r|r|r|}
		\hline
		Servicio & Precio mes & Meses & Total\\
		\hline
		Servidor privado virtual & 16,99\euro/mes & 11 & 186,89\euro\\
		\hline
		\hline
		\multicolumn{3}{|l|}{Total} & 186,89\euro\\
		\hline
	\end{tabular}
	\caption{Coste de los servicios subcontratados}\label{tab:serv_subcont}
\end{table}

\section{Recursos materiales empleados}

Durante la realización de este proyecto se ha hecho uso de una gran cantidad de software. Al haber sido software libre y gratuito el coste repercutido a un posible cliente es de 0\euro{} y no hace falta amortizarlo lo que puede ser considerado una ventaja importante frente a otros sistemas que pueden encarecer significativamente el producto.

Por lo anterior, solo es necesario repercutir el coste de los dispositivos materiales empleados, como puede verse en el cuadro~\ref{tab:re_mat}.

\begin{table}
	\centering
	
	\begin{tabular}{|l|r|r|}
		\hline
		Recurso & Cantidad & Coste total \\
		\hline
		Tesla T1070 & 1 & 7.082,00\euro \\
		\hline
		Servidor Tesla & 1 & 2.528,00\euro\\
		\hline
		\hline
		\multicolumn{2}{|l|}{Total} & 9.610,00\euro\\
		\hline
	\end{tabular}
	\caption{Coste de los recursos materiales empleados}\label{tab:re_mat}
\end{table}

\section{Amortizaciones}

La amortización es la perdida de valor del equipo por su uso y su antigüedad. Por este motivo debe ser repercutida en el proyecto y que así no se pierda dinero por la compra de los mismos.

Se ha considerado que un mes tiene en torno a las 160 horas laborables (4 semanas de 40 horas cada una) para la realización del cuadro~\ref{tab:amortizacion}.

\begin{table}
	\centering
	
	\begin{tabular}{|l|r|r|r|r|}
		\hline
		Recurso & Precio & Amortización (meses) & Uso & Repercutido\\
		\hline
		MacBook Pro & 2.600,00\euro & 60 & 60\% & 42,05\euro \\
		\hline
		Sobremesa & 1.200,00\euro & 60 & 40\% & 28,75\euro \\
		\hline
		\hline
		\multicolumn{4}{|l|}{Total} & 70,80\euro\\
		\hline
	\end{tabular}
	\caption{Coste por amortización de equipos}\label{tab:amortizacion}
\end{table}

\section{Gastos indirectos}

Se debe tener en cuenta el coste de todos aquellos elementos que, si bien no repercuten directamente sobre el proyecto, suponen un gasto constante durante la vida de éste.

\begin{table}
	\centering
	
	\begin{tabular}{|l|r|r|}
		\hline
		Descripción & Coste \\
		\hline
		Conexión a internet & 143,50\euro \\
		\hline
		Llamadas telefónicas & 143,50\euro \\
		\hline
		Material de oficina & 10,00\euro\\
		\hline
		Alquiler local & 3.953,13\euro\\
		\hline
		Electricidad & 74,75\euro\\
		\hline
		Servicio de limpieza & 3.593,75\euro\\
		
		\hline
		\hline
		Total & 7.918,63\euro\\
		\hline
	\end{tabular}
	\caption{Costes indirectos}\label{tab:cost_indi}
\end{table}

El precio de la conexión a internet y de la telefonía se ha calculado a partir de las facturas obtenidas de unos 40\euro/mes cada una. Considerando que un mes tiene 4 semanas a 40 horas laborables el precio por hora es de 0,25\euro/hora. Este valor se multiplica por el total de horas empleadas para realizar el proyecto.

El material de oficina comprende folios, cuadernos e instrumental para escribir. Apenas se ha gastado material de este tipo (un cuaderno, algunos folios y un bolígrafo) por lo que su valor es bajo.

El alquiler de locales en Leganés están sobre los 1.100\euro/mes de media (obtenido a partir de una revisión de \url{www.idealista.com}). Estos locales sólo pueden amortizarse durante la actividad económica del mismo (40h/semana). Esto supone un coste por hora de 6,875\euro/hora.

Se ha supuesto un gasto en electricidad de unos 20\euro/mes. Esto supone un coste de 0,13\euro/hora.

La limpieza del local puede rondar en torno a los 1.000\euro/mes. Se ha considerado que la subcontratación del servicio de limpieza puede rondar los 25\euro/hora y no sería necesario más de 2 horas al día durante cada día de la semana (unos 20 días al mes). Esto supone un coste de 6,25\euro/hora.

En el cuadro~\ref{tab:cost_indi} hay un resumen de los gastos indirectos del proyecto.

\section{Resumen del presupuesto}

\begin{table}
	\centering
	
	\begin{tabular}{|l|r|}
		\hline
		Descripción  & Coste  \\
		\hline
		Personal     & 9.200,00\euro \\
		\hline
		Subcontratas & 186,89\euro \\
		\hline
		Amortizaciones & 70,80\euro\\
		\hline
		Material     & 9.610.00\euro\\
		\hline
		Indirectos   & 7.918,63\euro\\
		\hline
		\hline
		Total        & 26.986,32\euro\\
		\hline
	\end{tabular}
	\caption{Resumen de los gastos del proyecto}\label{tab:resu_gastos}
\end{table}

Con todos los datos juntos (ver cuadro~\ref{tab:resu_gastos}) procedemos a calcular el precio de venta de la solución desarrollada.

Lo primero de todo es que pueden darse imprevistos a lo largo del desarrollo, como roturas de material, enfermedad del personal, etc. Por este motivo se aplica un margen del 10\% sobre el precio para compensar estos posibles imprevistos: 2.698,63\euro.

$$
\mbox{Coste Total} + \mbox{Margen de imprevistos} = \mbox{29.684,95\euro}
$$

El margen de beneficio a aplicar es del 25\% sobre el precio con imprevistos: 7.421,24\euro.

$$
	\mbox{Precio Final} = \mbox{Precio con Imprevistos} + \mbox{Beneficio} = \mbox{37.106,19\euro}
$$

El precio final de la solución, contando el 18\% de I.V.A., es: {\large \textbf{43.785,30\euro}}.